%%%%%%%%%%%%%%%%%%%%%%% file template.tex %%%%%%%%%%%%%%%%%%%%%%%%%
%
% This is a general template file for the LaTeX package SVJour3
% for Springer journals.          Springer Heidelberg 2010/09/16
%
% Copy it to a new file with a new name and use it as the basis
% for your article. Delete % signs as needed.
%
% This template includes a few options for different layouts and
% content for various journals. Please consult a previous issue of
% your journal as needed.
%
%%%%%%%%%%%%%%%%%%%%%%%%%%%%%%%%%%%%%%%%%%%%%%%%%%%%%%%%%%%%%%%%%%%
%
% First comes an example EPS file -- just ignore it and
% proceed on the \documentclass line
% your LaTeX will extract the file if required
\begin{filecontents*}{example.eps}
%!PS-Adobe-3.0 EPSF-3.0
%%BoundingBox: 19 19 221 221
%%CreationDate: Mon Sep 29 1997
%%Creator: programmed by hand (JK)
%%EndComments
gsave
newpath
  20 20 moveto
  20 220 lineto
  220 220 lineto
  220 20 lineto
closepath
2 setlinewidth
gsave
  .4 setgray fill
grestore
stroke
grestore
\end{filecontents*}
%
\RequirePackage{fix-cm}
%
%\documentclass{svjour3}                     % onecolumn (standard format)
%\documentclass[smallcondensed]{svjour3}     % onecolumn (ditto)
%\documentclass[smallextended]{svjour3}       % onecolumn (second format)
\documentclass[twocolumn]{svjour3}          % twocolumn
%
\smartqed  % flush right qed marks, e.g. at end of proof
%
\usepackage{graphicx}
\usepackage{amssymb}
\setcounter{tocdepth}{3}
\usepackage{listings}
\usepackage{subfigure}
\usepackage{mathtools}
\usepackage{url}
\usepackage{cite}
\urldef{\mailsa}\path|{jbrunelle, mkelly, hany, mweigle, mln}@cs.odu.edu|    
%\newcommand{\keywords}[1]{\par\addvspace\baselineskip\noindent\keywordname\enspace\ignorespaces#1}
\usepackage{colortbl}
\usepackage{color}
\usepackage{alltt}
\usepackage{wrapfig}
\usepackage{amsmath} % Added by MAT 20130711
%\usepackage[noend]{algpseudocode} % Added by MAT 20130711
%\usepackage[linewidth=1pt]{mdframed}
\usepackage[numbers]{natbib}

\hyphenation{Web-Cite}
\hyphenation{Java-Script}
\hyphenation{local-host}
\hyphenation{Arch-ive-it}
\hyphenation{Arch-ive-It}
\hyphenation{Ar-chive}
\hyphenation{Time-Map}
\hyphenation{Time-Maps}
\hyphenation{Mem-ento}
\hyphenation{Date-time}
\hyphenation{Date-times}
\hyphenation{date-time}
\hyphenation{date-times}
\hyphenation{style-sheet}
\hyphenation{Style-sheet}
\hyphenation{style-sheets}
\hyphenation{Style-sheets}
\hyphenation{Mem-ento-Date-times}
\hyphenation{Mem-ento-Date-time}

%
% \usepackage{mathptmx}      % use Times fonts if available on your TeX system
%
% insert here the call for the packages your document requires
%\usepackage{latexsym}
% etc.
%
% please place your own definitions here and don't use \def but
% \newcommand{}{}
%
% Insert the name of "your journal" with
% \journalname{myjournal}
%
\begin{document}

\title{Not All Mementos Are Created Equal: Measuring The Impact Of Missing Resources}

%\titlerunning{Short form of title}        % if too long for running head

\author{Justin F. Brunelle, Mat Kelly, Hany SalahEldeen, Michele C. Weigle, and Michael L. Nelson}

% the affiliations are given next; do not give your e-mail address
% unless you accept that it will be published
\institute{Old Dominion University, Department of Computer Science\\
Norfolk VA, 23529, USA\\
\mailsa\\}
\date{Received: date / Accepted: date}
% The correct dates will be entered by the editor


\maketitle
\begin{abstract}

Web archives do not always capture every resource on every page that they attempt to archive. This results in archived pages missing a portion of their embedded resources. These embedded resources have varying historic, utility, and importance values. The proportion of missing embedded resources does not provide an accurate measure of their impact on the Web page; some embedded resources are more important to the utility of a page than others. We propose a method to measure the relative value of embedded resources and assign a damage rating to archived pages as a way to evaluate archival success. In this paper, we show that Web users' perceptions of damage are not accurately estimated by the proportion of missing embedded resources. In fact, the proportion of missing embedded resources is a less accurate estimate of resource damage than a random selection. We propose a damage rating algorithm that provides closer alignment to Web user perception, providing an overall improved agreement with users on memento damage by 17\% and an improvement by 51\% if the mementos have a damage rating delta $\textgreater 0.30$. We use our algorithm to measure damage in the Internet Archive, showing that it is getting better at mitigating damage over time (going from a damage rating of 0.16 in 1998 to 0.13 in 2013). However, we show that a greater number of important embedded resources (2.05 per memento on average) are missing over time. Alternatively, the damage in WebCite is increasing over time (going from 0.375 in 2007 to 0.475 in 2014) while the missing embedded resources remains constant (13\% of the resources are missing on average). Finally, we investigate the impact of JavaScript on the damage of the archives, showing that a crawler that can archive JavaScript dependent representations will reduce memento damage by 13.5\%.

\keywords{Web Architecture, Web Archiving, Digital Preservation, Memento Damage}
\end{abstract}


\section{Introduction}
%\section[This is a very long title i want to break it manually]{This is a very long title\\ i want to break it manually}
\label{introduction}
Web archives are valuable cultural repositories that capture and store Web content. People (and robots) use archives like the Internet Archive \cite{iawebarchive, waybackarchives2} to retrieve archived material \cite{usingIA, marshalls_social_media_study} for a variety of purposes and in a variety of ways \cite{yasminLinks}. However, the resources being requested by Web users may not be complete; embedded resources are sometimes missing from an archived Web page \cite{ipresArchivability}. Missing embedded resources return a non-200 HTTP status (e.g., 404, 503) when their URI is dereferenced.

Archivists work to ensure archives are as complete -- and as high quality -- as possible. Through identifying sources of missing content or archival difficulties, archivists can address archival challenges by taking steps to adjust processes or to fill in gaps in archive collections.

Reyes et al. identified current efforts within several archives to assess their archival collections \cite{archiveQA}. Of the archivists sampled 61\% confirmed that their goal is to assess the quality of every Web page captured, 43\% assess quality and success using a simple  boolean or numerical notion of completeness based upon the number of missing embedded resources in the Web pages. As we will demonstrate in this paper, human perception of quality is not accurately represented with a measure of the proportion of missing embedded resources. For example, large images are often more important to an archived page's utility than small images. Similarly, style sheets that format visible content are more important to the representation of the page than style sheets without significant formatting responsibilities. We provide a mechanism to assess the impact of missing embedded resources in the archives that improves upon simply measuring the percent of missing embedded resources.

Of the archivists surveyed by Reyes et al. that conduct quality assurance, 100\% use a manual process. The Internet Archive alone boasts 455 billion web pages in its archive\footnote{According to the text at \url{https://archive.org/web/} at the time of authoring}, which is far larger than can be evaluated through human methods. While Banos et al. constructed the CLEAR method to assign a predictive archivability score \cite{ipresArchivability}, a similar score for the actual performance of an archival tool does not exist outside of the simple metric of the percent of embedded resources archived. 
An algorithm to automatically assess human perception of archived page quality would significantly decrease the necessary human involvement in the quality assurance process, potentially increasing the accuracy while reducing the cost of quality assurance efforts.


Throughout this paper we use Memento Framework terminology. Memento \cite{nelson:memento:tr} is a framework that standardizes Web archive access and terminology. Original (or live web) resources are identified by URI-R, and archived versions of URI-Rs are called \emph{mementos} and are identified by URI-M. Memento TimeMaps are machine-readable lists of mementos (at the level of single-archives or aggregation-of-archives) sorted by archival date.

This research has three goals. First, we want to understand how missing embedded resources impacts Web users' perceived quality of a memento. Using an algorithm to measure embedded resource importance, we determine whether an important embedded resource of the memento is missing (e.g., a main image or video essential to the user's understanding of the page), or if the missing embedded resource contributes little to the memento's utility for the user (such as a spacer image or small logo). We propose a method of weighting embedded resources in a memento according to importance ($D_m$). We show that $D_m$ is an improved damage rating over an unweighted proportion of missing embedded resources to all requested resources ($M_m$). We use Amazon's Mechanical Turk to compare our algorithm to Web users' notion of damage and to show an improvement over the unweighted count of missing embedded resources.

Second, we use our algorithm to assess the damage of mementos in the Internet Archive and WebCite. We compare the $M_m$ and $D_m$ based on Web user agreement with the metrics. %Using this proportion of missing embedded resources as an unweighted comparison baseline, we use our weighting algorithm to determine memento damage in the Internet Archive and use Mechanical Turk to compare the accuracy of each algorithm.

Third and finally, we measure damage in the Internet Archive and WebCite over time using $D_m$. We describe how this algorithm can be used for future enhancements of the Heritrix crawler \cite{Sigurosson:Incremental-Heritrix, heritrix} and Internet Archive's archival processes. We also discuss the impacts of JavaScript on archive quality, using WebCite as the target of our discussion, and compare WebCite's memento quality to Archive.today.

%In this paper, we examine the results of our work developing an algorithm to accurately measure memento damage (Section \ref{damage}), use Mechanical Turk to evaluate our algorithm against Web user interpretation of damage (Section \ref{turk}), and measure the quality of mementos in the Internet Archive using our algorithm (Section \ref{eval}). 


\section{Motivating Examples}
\label{example}


\begin{figure*}[h!t]
  \begin{center}
    \subfigure[All three of the embedded images are included in $m_0$ and identified by the red arrows (\emph{$M_m$=0.17}).]{\label{undamaged}\includegraphics[width=0.3\textwidth]{./imgs/undamagedAnnotated.png}}\qquad
    \subfigure[We removed the large, central image (that is the main content of the page) from $m_1$, identified by the red arrow (\emph{$M_m$=0.24}).]{\label{missingBig}\includegraphics[width=0.3\textwidth]{./imgs/missingBigAnnotated.png}}  \qquad
    \subfigure[We removed the XKCD logo and banner of comics from $m_2$, identified by the red arrows (\emph{$M_m$=0.29}).]{\label{missingLittle}\includegraphics[width=0.3\textwidth]{./imgs/missingLittleAnnotated.png}}  \\
    
    \subfigure[This memento (URI-M \protect\url{http://web.archive.org/web/20110116022653/http://www.cityofmoorhead.com/flood/?}) is missing a single style sheet which changes the entire appearance and utility of the memento (\emph{$M_m$=0.38}).]{\label{missingflood}\includegraphics[width=0.45\textwidth]{./imgs/missing_flood_page_crop.png}}\qquad
%,height=8cm,keepaspectratio
    \subfigure[Meanwhile, this memento (URI-M \protect\url{http://web.archive.org/web/20060102083228/http://www.ascc.edu/}) is missing two style sheets (along with two images) but does not appear damaged (\emph{$M_m$=0.20}).]{\label{missingaldot}\includegraphics[width=0.45\textwidth]{./imgs/missing_lots_crop.png}}  
  \end{center}
  \caption{Mementos have different meanings and usefulness depending on which embedded resources are missing from the memento (and the proportion of missing resources, $M_m$).}
  \label{xkcdImgs}
\end{figure*}

We use the XKCD Web page as an example of a resource with embedded resources of differing importance. We captured the URI-R using the wget \cite{wget} command\footnote{We executed the wget command with parameters as follows: \texttt{wget -E -H -k -K -p} \url{http://www.xkcd.com/}} and manually inflicted damage on a local memento of \url{http://www.xkcd.com/} by removing embedded images. We used PhantomJS \cite{pjs} to dereference the URI-M and take a PNG snapshot of the representation, and we recorded the resulting HTTP response headers of the embedded resources. We created three mementos of the URI-R: one duplicating its live Web counterpart 
%(we call this the \emph{original} memento)
($m_0$), one with the central comic image removed ($m_1$), and one with two logo images removed ($m_2$). The snapshots taken by PhantomJS are provided in Figures \ref{undamaged}, \ref{missingBig}, and \ref{missingLittle}. As shown in the captions, the proportion of embedded missing resources to all requested resources ($M_m$) varies among the mementos.

At the time of this test, the live XKCD site was missing two embedded style sheets, as are $m_0$, $m_1$, and $m_2$ since they are copies of the live site. We verified that our memento $m_0$ has a $M_m$ value identical to its live Web counterpart -- the live resource and $m_0$ are both missing the same embedded resources ($M_m$=0.17). In Figure \ref{undamaged}, $m_0$ has multiple embedded resources, but we focus on the three identified by the red arrows: the XKCD logo, the main comic image, and the banner of comics. The central image is most important to the utility of the page -- without the main comic image, the user does not obtain the information from the page that the author intended (Figure \ref{missingBig}). The logo and banner are not essential to the user's understanding of the XKCD content (Figure \ref{missingLittle}). 

Cascading Style Sheets (CSS) also differ in importance. Some style sheets are responsible for formatting small portions of a page, while others are responsible for placing images and other content or even organizing the entire page for the user. Figure \ref{missingflood} shows a memento of a URI-R that is missing a single style sheet. This style sheet is responsible for a large amount of information in the representation and without it, the meaning and utility of the memento changes. Figure \ref{missingaldot} shows a memento that is properly styled but is missing two style sheets that are not responsible for the majority of the content organization.% and the memento is still properly styled without them.

As we have discussed, the percentage of successfully dereferenced embedded resources is not the only factor in determining memento quality. In support of that principle, we refer to Figure \ref{missingaldot} in which $M_m$=0.2 (6/30). However, it appears to be well-preserved. In our XKCD example, Figure \ref{missingLittle} is missing two images ($M_m$=0.29) yet maintains more important embedded mementos than Figure \ref{missingBig} ($M_m$=0.24). These examples support the motivation of our research by demonstrating that unweighted percentages (i.e., $M_m$) are insufficient to assesses perceived memento damage.


\section{Related Work}
\label{priorwork}
Researchers have studied the completeness of the archives, the recrawl policies that optimize archive quality, and the relative importance of content within Web resources. We build upon these prior works and apply their findings to develop our algorithm for automatically assessing the quality of mementos.

SalahEldeen et al. have studied the rate at which live resources disappear from the Web. In a study of the Egyptian Revolution, SalahEldeen found that 11\% of the resources shared over Twitter were missing after one year \cite{losingmyrevolution, hanyTPDL2013}. %Studies such as SalahEldeen's show that resources are disappearing from the Web and are inadequately archived. We are investigating how institutions can improve archival quality by assessing the quality of mementos to determine if mementos are adequately archived. 

Our previous work studied the factors influencing archivability, including accessibility standards and their impact on memento completeness \cite{kellyTPDL2013}. In this work, we used a yearly sampling method to select mementos for testing. We use a similar method in this work to study memento damage.

Spaniol et al. measured the quality of Web archives based on matching crawler strategies with resource change rates and related implications for crawling strategies \cite{spaniol9catch, spaniol2009data, Denev:2009:SFQ:1687627.1687694}. Ben Saad and Gan\c{c}arski performed a similar study regarding the importance of changes on a page \cite{saad2011, saadIJDL, saadTPDL}. Gray and Martin created a framework for high quality mementos and assessing their quality by measuring the missing embedded resources \cite{mementoQuality}. While these studies focused on memento completeness and site coverage, we focus on assessing the importance of the artifacts that are missing. 

Banos et al. created the CLEAR algorithm to evaluate archival success based on adherence to standards for the purpose of assigning a resource archivability score \cite{ipresArchivability}. The authors expanded on CLEAR and created CLEAR+ in their follow-on efforts \cite{clearIJDL}. 

Fersini et al. studied the importance of information blocks of a rendered Web page, finding that blocks with more images are more important \cite{Fersini20081431}. Singh et al. found that multimedia within a page is essential for user understanding \cite{Singh2009}. Ye et al. found that the information blocks close to the center of the viewport contain important information, while ``noise'' -- or unimportant content -- occurs on the fringes or edges of the page \cite{Yi2003}. Kohlsch\"{u}tter et al. also found that important content was located in the center of pages \cite{boilerPlate}. Centrality is a way for authors to convey importance of information to their users. For example, images in the center of the viewport are more important or contribute more to the users' understanding of a page than those positions on the fringes or outside the viewport of a page. Using these prior findings, we constructed an algorithm to assess the importance of embedded resources based on their file type, location in the viewport, and size in pixels.

Zhang et al. studied human perception and human ability to recognize differences in images effectively determining human perception limitations for images at the pixel level \cite{Zhang200830}. Rademacher et al. used human perception to identify the visual factors that distinguished computer generated images from photographs \cite{rademacher}. We use human perception in a similar way to identify levels of memento damage.

The algorithm proposed in this paper determines the importance of embedded resources. Song et al. outlined an algorithm for determining the importance of sections of Web pages based on their content, size, and position \cite{blockImportance}. Song's work focused on recognizing important blocks of a Web page to eliminate noise in an effort to accurately extract aspects of pages that users would find most important. Blocks featured prominently in the center of the view port and occupying a large area of the page were found to be most important. We utilize this concept, identifying content occupying large amounts of viewport real-estate as important in our measurements of the importance of missing embedded resources.



\section{Users' Perception of Damage}
\label{turk} 

%The damage rating we defined in Section \ref{damage} was constructed from the point of Web archivists; our notions of importance were imbued in the algorithm. Despite supporting evidence (e.g., prior works support our notion that resources centered in the viewport are more important than those on the fringes \cite{Yi2003, boilerPlate}), we needed to validate that our measure of damage matches Web users' interpretation of damage. We used Amazon's Mechanical Turk to evaluate the $D_m$.

As archivists, our perception of damage differs from that of more traditional Web users. To determine if $M_m$ (percent missing) is a good estimate of human perception of damage, we used Amazon's Mechanical Turk to measure human agreement with $M_m$.

To ensure that Mechanical Turk workers (or more colloquially, ``turkers'') could evaluate damage, we presented turkers with pairs of mementos that had varying levels of damage and asked them to select the memento they preferred to keep if given a choice between the two.

We captured 11 hand-selected URI-Rs (Table 1) on a local server and created five versions of the mementos for each URI-R. We manually damaged the mementos to create the five categories of damage. For the category \emph{missing image}, we removed a prominent image (empirically identified as important) from the memento. For the category \emph{missing css}, we removed a prominent CSS file to cause formatting issues in the memento; we empirically selected the CSS file to remove based on the greatest human-perceived detrimental impact to the page layout. We also created the categories \emph{missing all images} (we removed every embedded image), \emph{missing all resources} (we removed all embedded resources), and \emph{original} (the URI-M was a direct copy of the live resource) and measured the $M_m$ of each URI-M in each category. We refer to the four categories of damaged mementos in aggregate as $m_1$ and the \emph{original} as $m_0$. These categories created several degrees of damage through a variety of missing embedded resources for identical URI-Rs at an identical time point to provide a wide spectrum of mementos to be evaluated by turkers.

\begin{table*}[h!t]
\begin{tabular}{ p{6cm} | c |  c |  c |  c |  c }
	 & \multicolumn{5}{c}{$M_m$}\\
    URI-R & $m_0$ & missing image & missing css & missing all images & missing all\\
    	\hline
    	\hline
    \protect\url{http://www.cs.odu.edu/~mln/} & 0.14 & 0.43 & 0.29 & 0.43 & 0.43 \\
    	\hline 
    \url{http://activehistory.ca/2013/06/myspace-is-cool-again-too} \url{-bad-they-destroyed-history-} \url{along-the-way/comment-page-1/} & 0.0 & 0.32  & 0.32 & 0.57 & 0.85 \\
    	\hline
    \protect\url{http://www.albop.com/} & 0.0 & 0.13 & 0.0 & 0.50 & 0.50\\
    	\hline
    \protect\url{http://www.cs.odu.edu/} & 0.10 & 0.13 & 0.11 & 0.82 & 0.81 \\
    	\hline
    \protect\url{http://ws-dl.blogspot.com/2013/08/2013-07-26-web-archiving-and-digital.html} & 0.07 & 0.08 & 0.08 & 0.13 & 0.14 \\
    	\hline
    \protect\url{http://www.cnn.com/2013/08/19/tech/social-media/zuckerberg-facebook-hack/} & 0.19 & 0.22 & 0.28 & 0.46 & 0.57 \\
    	\hline
    \protect\url{http://xkcd.com/} & 0.14 & 0.38 & 0.31 & 0.53 & 0.54 \\
    	\hline
    \protect\url{http://www.mozilla.org/} & 0.80 & 0.80 & 0.80 & 0.877 & 0.89 \\
    	\hline
    \protect\url{http://www.ehow.com/} & 0.05 & 0.05 & 0.06 & 0.11 & 0.33 \\
    	\hline
    \protect\url{http://google.com/}  & 0.0 & 0.0 & 0.0 & 0.0 & 1.0  \\
    	\hline
    \protect\url{http://php.net/} & 0.32 & 0.33 & 0.33 & 0.37 & 0.37  \\
    	\hline
\end{tabular}
  \caption{The 11 URI-Rs used to create the manually damaged dataset. $M_m$ values are provided for each $m_1$.}
  \label{dataset}
\end{table*}


\begin{figure}[h!]
\includegraphics[width=0.50\textwidth]{./imgs/turkss.png}
\caption{We asked the turkers to select the less damaged of two mementos. The two versions of the page are accessible in separate tabs.}
\label{turkss}
\end{figure}


With the goal of determining whether or not turkers can recognize damage in a memento, we presented the turkers with an $m_1$ and its $m_0$ counterpart (that is, a ``damaged'' and its \emph{ground-truth} memento) and asked the turkers ``We saved two versions of the same website ... \emph{Which version did we do a better job saving?}'' (Figure \ref{turkss}). For each URI-R, a pair of mementos consisting of $m_0$ and one of the four categories of $m_1$ were evaluated by five turkers for a total of 280 evaluations. We follow the precedent of using five turkers to establish turker opinion as established by SalahEldeen and Nelson \cite{hanyTurk}. 

We show the judgement splits from the turker evaluations in Table \ref{m0table}. The judgement splits refer to the number of turkers that selected the correct-incorrect version. For example, a 0-5 split means all five turkers selected the $m_1$ (an incorrect selection), a 5-0 split means all five turkers selected the $m_0$ memento (the correct selection), and a 3-2 split means three turkers selected the $m_0$ memento and two selected the $m_1$ (a correct selection by the majority, but still a split decision among the turkers). For the purposes of this paper, we consider only 5-0 and 4-1 splits as agreement with $M_m$ and all other splits as disagreement. {$\Delta M_m$} refers to the delta between $M_{m_0}$ and $M_{m_1}$.

The turkers selected $m_0$ as the preferred option (less damaged memento) 81\% of the time (226/280). As {$\Delta M_m$} grows, turker agreement is more consistent. 


\begin{table}
\begin{tabular}{ c | c | c | c | c | c | c || c}
    {$\Delta M_m$} &  \multicolumn{6}{c}{Splits}\\
  & 5-0 & 4-1 & 3-2 & 2-3 & 1-4 & 0-5 & Total\\
\hline
1.0   &  &  &  &  &  & & 0.00\\
0.9 &  &  &  &  &  & & 0.00\\
0.8 & 4 &  &  &  &  & & 0.07\\
0.7 &  &  &  &  &  & & 0.00\\
0.6 &  &  &  &  &  & & 0.00\\
0.5 & 1 & 1 &  &  &  & & 0.04\\
0.4 &  &  &  &  &  & & 0.00\\
0.3 & 15 & 5 &  &  &  & & 0.36\\
0.2 & 2 &  &  &  &  & & 0.04\\
0.1 & 5 & 4 & 4 & 2 &  & 1& 0.29\\
0.0 & 5 & 3 & 1 & 3 &  & & 0.22\\
\hline
Total & 0.58 & 0.23 & 0.09 & 0.09 & 0.00 & 0.02 & 1.0
\end{tabular}
  \caption{The turkers selected $m_0$ as the preferred memento 81\% of the time, and more consistently for larger {$\Delta M_m$} values.}
  \label{m0table}
\end{table}

%\begin{table}
%\centering
%\begin{tabular}{c >{\bfseries}r @{\hspace{0.7em}}c @{\hspace{0.4em}}c }
%  \multirow{10}{*}{\rotatebox{90}{\parbox{4cm}{\bfseries\centering Turker Assessment}}}
%    & \multicolumn{3}{c}{\bfseries $M_m$}\\
%  & & \bfseries Select $m_0$ & \bfseries Select $m_1$\\
%  & $m_0$ & \MyBox{44}{ } & \MyBox{0}{ }\\[2.4em]
%  & $m_1$ & \MyBox{11}{ } & \MyBox{0}{ }\\
%\end{tabular}
%  \caption{Confusion matrix of the turker assessments of the $m_0$ vs $m_1$ comparison test.}
%  \label{m0cm}
%\end{table}

\begin{table}
\centering\begin{tabular}{cll}
\textbf{Turker} & \multicolumn{2}{c}{ \textbf{ $M_{m}$}}                           \\
\textbf{Assesment}                  &          Select $m_{0}$             &           Select $m_{1}$             \\ \cline{2-3} 
       $m_{0}$           & \multicolumn{1}{|l}{44} & \multicolumn{1}{|l|}{0} \\ \cline{2-3} 
       $m_{1}$           & \multicolumn{1}{|l}{11} & \multicolumn{1}{|l|}{0} \\ \cline{2-3} 
\end{tabular}
 \caption{Confusion matrix of the turker assessments of the $m_0$ vs $m_1$ comparison test.}
  \label{m0cm}
\end{table}


Regardless of {$\Delta M_m$}, 81\% of the evaluations agreed with $M_m$ as a suitable damage metric (5-0 and 4-1 splits). Turkers were unsure about the damage (3-2 and 2-3 splits) 18\% of the time and incorrectly identified damage only once. The average {$\Delta M_m$} for the unsure selections was $\textless$ 0.01, and the only 0-5 split had a {$\Delta M_m$} of 0.014, suggesting that confusion or disagreement occurs more often when the damage delta is smaller. 

Confusion matrices provide a consolidated view of an algorithm's performance. The top left quadrant shows the number of true positives, the top right shows the number of false negatives, the bottom left shows false positives, and the bottom right shows true negatives. The algorithm's accuracy ((True Positives + True Negatives) / (All Positives and Negatives)) and harmonic mean (or $F_1$ Score: 2 $\times$ True Positives / (2 $\times$ True Positives + False Positives + False Negatives)) are calculated using a confusion  matrix. A harmonic mean provides an average (in this case, of the algorithm's success rate) and is sensitive to small values and outliers.  

From the confusion matrix (Table \ref{m0cm}), we can calculate $F_1$=0.88 and accuracy=0.80 for $m_0$ vs $m_1$. Turker agreement does not match $M_m$ 100\% of the time with the $m_0$ vs $m_1$ test because of phenomena with aesthetics and human perception. Also, $m_0$ is often incomplete ($M_{m_0} \textgreater 0$) and, as a result, has $D_{m_0} \textgreater$ 0\footnote{Live Web resources may have missing embedded resources, and this results in a calculated $D_{m_0} \textgreater$ 0.} (see php.net in Table \ref{dataset}).

Ideally, turker agreement would be unanimous. The measured turker agreement of 81\% can be attributed to one of several factors. The {$\Delta M_m$} measures were very small in comparisons resulting in split turker decisions, potentially causing the damaged mementos to look very similar based on human perception. This phenomenon further illustrates the need for a $D_m$ measure because turkers have dissenting opinions when mementos are damaged in visually similar ways. Regardless of the reason the turker agreement fell short of 100\% with $M_m$, we demonstrate an improvement of $D_m$ over $M_m$.

\section{Evaluating Organic Damage}
\label{turkActual}
Having identified $m_0$ in the $m_0$ vs $m_1$ in a large majority (81\%) of the comparisons, the turkers have shown that they can identify a damaged resource when presented a damaged and undamaged memento. Because they can identify damage in mementos, we used turkers to evaluate our measured damage of mementos found in the Internet Archive. 

\subsection{Dataset Selection}
\label{datasetSelection}

This experiment uses the same set of 2,000 URI-Rs as in our previous work \cite{ijdl}, which was sampled from Twitter and Archive-It. The first dataset, the \emph{Twitter} set, consists of 1,000 Bitly URIs shared over Twitter and represents a more random selection of URI-Rs not explicitly selected for curation by human archivists. We collected the Twitter URIs through the Twitter Garden Hose\footnote{\url{https://dev.twitter.com/docs/streaming-apis/streams/public}} in October 2012. 

The second dataset, the \emph{Archive-It} set, was sampled from Archive-It collections. Archive-It collections are created and curated by human archivists often corresponding to a certain event (e.g., National September 11 Memorial Museum) or a specific set of Web sites (e.g., City of San Francisco). The Archive-It set consists of the entire set of URI-Rs belonging to the collections listed on the first page of collections at Archive-It.org\footnote{\url{http://www.archive-it.org/explore/?show=Collections}} as of October 2012. This resulted in 2,093 URI-Rs that represent a collection of previously archived and human-curated URIs. To make the datasets equal in size, we randomly sampled 1,000 URI-Rs from the set of 2,093. 

We discarded non-HTML representations (e.g., JPEG and PDF) from both sets and combined the Twitter and Archive-It datasets for a final dataset of 1,861 URI-Rs. Non-HTML representations do not contribute to this study since they do not have embedded resources. There is no overlap between the two sets.

As measured in our prior work \cite{ijdl}, the resources in the Archive-It set receive an HTTP 200 response for 93.5\% of all requests for embedded resources and the resources in the Twitter set receive an HTTP 200 response for 87.1\% of all requests for embedded resources.

\subsection{Turker Evaluation}
\label{turkEval}

Using this set of URI-Rs, we measured the damage of one memento per year from the Internet Archive TimeMap of each of the 1,861 URI-Rs, resulting in 45,341 URI-Ms. We randomly selected a subset of 100 URI-Ms from this set. Similar to the evaluation in Section \ref{turk}, we gave turkers two mementos (we will generalize these to $m_2$ and $m_3$) from consecutive years from the same TimeMap and asked the turkers to select the less damaged memento (``We saved two versions of the same website ... \emph{Which version did we do a better job saving?}'') as shown in Figure \ref{turkss}. Because $m_2$ and $m_3$ are observed from the Internet Archive, neither is considered a \emph{ground-truth}. We measured $M_m$ of mementos in the Internet Archive and compared it to the turker perception of the utility of the mementos. 

Contrary to the test in Section \ref{turk}, as {$\Delta M_m$} grows, the turkers are not as effective at selecting the less damaged memento (the splits are shown in Table \ref{m1table}). The turkers only agree with $M_m$ 12\% of the time and completely disagree with $M_m$ (1-4 and 0-5 splits) 44\% of the time. This discrepancy demonstrates that turker assessment of damage does not match $M_m$. Additionally, we see that the turkers performed well when comparing $m_0$ vs $m_1$ (original vs damaged) but struggle to compare $m_2$ vs $m_3$ (damaged vs damaged).


\begin{table}
\begin{tabular}{ c | c | c | c | c | c | c || c}
    {$\Delta M_m$} &  \multicolumn{6}{c}{Splits}\\
  & 5-0 & 4-1 & 3-2 & 2-3 & 1-4 & 0-5 & Total\\
\hline
1.0 &  &  &  &  & 1 & & 0.01\\
0.9 &  &  &  &  &  & & 0.00\\
0.8 &  &  &  &  &  & & 0.00\\
0.7 &  & 1 &  &  &  & & 0.01\\
0.6 &  &  &  &  & 1 & & 0.01\\
0.5 &  &  &  &  &  & & 0.00\\
0.4 &  & 1 &  &  &  & & 0.01\\
0.3 & 1 &  & 3 & 4 & 1 & 2& 0.11\\
0.2 &  & 5 & 6 & 5 & 12 & 9& 0.37\\
0.1 & 4 & 5 & 10 & 11 & 15 & 3& 0.48\\
0.0 &  &  &  &  &  & & 0.00\\
\hline
Total & 0.05 & 0.12 & 0.19 & 0.20 & 0.30 & 0.14 & 1.0
\end{tabular}
  \caption{The turker evaluations of the $m_2$ vs $m_3$ comparisons when using $M_m$ as a damage measurement.}
  \label{m1table}
\end{table}

\begin{table}
\centering
\begin{tabular}{cll}
\textbf{Turker} & \multicolumn{2}{c}{ \textbf{ $M_{m}$}}                           \\
\textbf{Assesment}                  &          Select $m_{2}$             &           Select $m_{3}$             \\ \cline{2-3} 
       $m_{2}$           & \multicolumn{1}{|l}{29} & \multicolumn{1}{|l|}{24} \\ \cline{2-3} 
       $m_{3}$           & \multicolumn{1}{|l}{23} & \multicolumn{1}{|l|}{24} \\ \cline{2-3} 
\end{tabular}
  \caption{Confusion matrix of the turker assessments of the $m_2$ vs $m_3$ comparison test against $M_m$.}
  \label{m1cm}
\end{table}

%acc = (TP+TN)/(P+N)
%acc = (29+17)/(29+24+23+24)
%f1 = 2tp/(2tp + fp + fn)
%f1 = (2*29)/(2*29 + 23 + 24) 


From the confusion matrix (Table \ref{m1cm}), we can calculate the accuracy of turker selections of $m_2$ vs $m_3$ agreement with $M_m$ is 0.46 with $F_1$=0.55. In a Receiver Operating Characteristic (ROC) curve \cite{fawcett2006introduction}, we calculated the Area Under the ROC Curve (AUC) for the results of the turker evaluations of $m_2$ vs $m_3$ against $M_m$ and the results of the manually damaged $m_0$ vs $m_1$ test. The AUC of $M_m$ is lower (AUC=0.472) than random (AUC=0.500) as shown in Table \ref{auc1}, meaning that $M_m$ performed worse than random for matching turker perception of damage.


\begin{table}
\centering
\begin{tabular}{ c | c | c | c }
    Damage Calculation &  AUC & $F_1$ & Accuracy\\
\hline
  $M_m$ & 0.472 & 0.55 & 0.46\\
  $M_{m_0}$ & 0.789 & 0.88 & 0.80 \\
\hline
\end{tabular}
  \caption{When compared to random, $M_m$ performs worse than random selection and is worse than the performance of $m_0$ vs $m_1$.}
  \label{auc1}
\end{table}



%%%%%%%%%%%%%%%%%%%%%%%%%%%%
%%%%%%%%%%%%%%%%%%%%%%%%%%%%
%%%%%%%%%%%%%%%%%%%%%%%%%%%%

\section{Calculating Memento Damage}
\label{damage}
With $M_m$ not matching Web users' perception of damage, we propose a new algorithm for assessing memento damage. Our proposed algorithm is based on the file type, size, and location of the embedded resource.

\subsection{Defining $D_m$ and $M_m$}
Before defining equations for our memento quality measurements, we first describe the resources in the mementos in Equation \ref{resources}, differentiating between the set of all embedded resources \emph{R} and the set of all missing resources \emph{$R_m$}. In this case, we consider any resource needed to build a resource and that is requested by the client an \emph{embedded resource}.

\begin{equation}
\label{resources}
\begin{split}
R &= \{\text{All embedded resources requested}\}\\
R_m &= \{\text{All missing embedded resources}\}\\
&R_m \subseteq R
\end{split}
\end{equation}

As we mention in Section \ref{example}, we calculate $R$ by counting the number of distinct and unique URIs requested by the client when dereferencing the URI-M. For example, if an image identified by $\textit{URI-R}_a$ is referenced three times in the DOM, it is only requested once by the client and is only counted once in $R$. Similarly, we calculate $R_m$ by counting only the URI-Rs that, when dereferenced, return an HTTP response code in the 400 or 500 range (i.e., is not successfully dereferenced). If an HTTP GET for $\textit{URI-R}_a$ returns an HTTP 404 response (or an HTTP 503 response), it counts once in $R_m$.

Our measurement of $M_m$ is the proportion of missing embedded resources to all requested resources (Equation \ref{missingeqn}). We define $M_m$ as a proportion because it normalizes the measurement. Without using a proportion, $M_m$ breaks down when mementos have a very large or very small number of embedded resources. For example, a memento with two embedded resources and is missing one has a lower archiving success rate than a memento with one hundred embedded resources and is missing one. Normalizing $M_m$ allows use to compare mementos that have different numbers of embedded resources using the same metric.

The $M_m$ measure includes resources that were omitted from a crawl due to crawl policies or robots.txt \cite{robotsProtocol} because the goal of $M_m$ and $D_m$ is to help identify damage independently of conscious efforts of the archival institutions. 

\begin{equation}
\label{missingeqn}
M_m = \frac{R_m}{R}
\end{equation}

We define $D_m$ as the damage rating, or cumulative damage, of a memento $m$ in Equation \ref{damageeqn}. $D_m$ is a normalized value ranging from $[0,1]$. We calculate the potential damage of a memento and the actual damage of a memento and express the damage rating as the ratio of actual to potential damage. Notionally, potential damage is the cumulative importance of all embedded resources in the memento, while actual damage is only the importance of those embedded resources that are unsuccessfully dereferenced, or missing. 

\begin{equation}
\label{damageeqn}
D_m = \frac{D_{m_{actual}}}{D_{m_{potential}}}
\end{equation}


\subsection{Weighting Embedded Resources}
We calculate the importance of each embedded resource in the set \emph{R}. The sum of each embedded resource is the potential damage \emph{$D_{m_{potential}}$} (Equation \ref{potentialdamage}). Important resources are assigned additional weights to increase their relative value over unimportant resources (Equations \ref{imagedamage} - \ref{cssdamage}).

\begin{equation}
\label{potentialdamage}
\begin{split}
D_{m_{potential}}& = \frac{\sum_{i=1}^{n_{[I,MM]}} D_{[I|MM]}(i)}{n_{[I|MM]}} +\frac{\sum_{i=1}^{n_C} D_C(i)}{n_C} %\frac{\sum_{i=1}^{n_{MM}} D_{MM}(i)}{n_{MM}}\\& + D_T(M)\\&\forall\{\text{I=Images, MM=Multimedia, C=CSS}\}\\& n \in R
\\&\forall\{\text{I=Images, MM=Multimedia, C=CSS}\}\\& n \in R
\end{split}
\end{equation}

Actual damage ($D_{m_{actual}}$, defined in Equation \ref{actualdamage}) is identical to $D_{m_{potential}}$ except it is computed using only the missing embedded resource set \emph{$R_m$}. %As shown in Equation \ref{damageeqn}, the damage rating is the ratio of actual damage to potential damage.

\begin{equation}
\label{actualdamage}
\begin{split}
D_{m_{actual}} =& \frac{\sum_{i=1}^{n_{[I,MM]}} D_{[I,MM]}(i)}{n_{[I|MM]}} + \frac{\sum_{i=1}^{n_C} D_C(i)}{n_C} %\frac{\sum_{i=1}^{n_{MM}} D_{MM}(i)}{n_{MM}}\\& + D_T(M)\\&\forall\{\text{I=Images, C=CSS, MM=MultiMedia}\} \\&n \in R_m
\\&\forall\{\text{I=Images, MM=Multimedia, C=CSS}\} \\&n \in R_m
\end{split}
\end{equation}


In $M_m$, as opposed to $D_m$, all embedded resources are considered equal. The potential damage is therefore the number of embedded resources, and the actual damage is the number of missing embedded resources. $M_m$ is the unweighted ratio of missing embedded resources to total embedded resources.

We introduce additional weights of differing values to account for the notion of embedded resource importance. When a weight $w$ is given to an embedded resource, all $n$ embedded resources lose $\frac{w}{n}$ importance, which redistributes the importance between embedded resources while keeping the sum of all importance constant. Note that we only assign additional weights to embedded resources that are visually validated as present (i.e., images, multimedia, and style sheets); the weighted importance of other embedded resources is considered outside of the scope of this research.

\subsubsection{Image Damage Calculation}
To account for image importance, images receive weights $w$ for image size and centrality (Equation \ref{imagedamage}). We use the pixel area (width $x$ height) of the image as specified in the HTML and the page size along with a weight for horizontal and vertical central dividing line overlap by the image. We omit the size attribute from the calculation if the image dimensions are missing from the HTML. For example, we can extract the width and height of the missing embedded resource ``IMAGE.png'' from this HTML

\begin{verbatim}
<img src="IMAGE.png" height="42" width="42">
\end{verbatim}

\noindent but not this HTML

\begin{verbatim}
<img src="IMAGE.png">.
\end{verbatim}



\begin{equation}
\label{imagedamage}
\begin{split}
&D_{[I|MM]} = 1 + \frac{width \times height}{\text{Page Size (pixels)}} + w_{\text{horizontal}} \\&+ w_{\text{vertical}}\\
&w_{\text{horizontal}} =  \left\{
  \begin{array}{l l}
    0.25 & \quad \text{image overlaps horizontal center}\\
    0    & \quad \text{otherwise}
  \end{array} \right.\\
&w_{\text{vertical}} =  \left\{
  \begin{array}{l l}
    0.25 & \quad \text{image overlaps vertical center}\\
    0    & \quad \text{otherwise}
  \end{array} \right.
\end{split}
\end{equation}

%&+ ({w_{centrality}}{2} \iff \text{Overlaps vertical middle})\\

Embedded multimedia importance ($D_{MM}$) is calculated identically to image importance $D_I$, and we represent both in the same equation $D_{[I|MM]}$. Because size and centrality determine multimedia importance, we omit audio and other non-visual multimedia resources. We also classify Flash movies as multimedia.

%\begin{equation}
%\label{mmdamage}
%\begin{split}
%D_{MM} &= 1 + \frac{width * height}{\text{Page Size}}\\ &+ (\frac{w_{centrality}}{2} \iff \text{Overlaps horizontal middle})\\
%&+ (\frac{w_{centrality}}{2} \iff \text{Overlaps vertical middle})\\
%w_{centrality} &= 0.50
%\end{split}
%\end{equation}

\subsubsection{Style sheet Damage Calculation}
Equation \ref{cssdamage} outlines the damage from missing style sheets, including a factor for a style threshold $w_{style}$ and a threshold for non-matching CSS tags in the DOM $w_{tags}$. 


\begin{equation}
\label{cssdamage}
\begin{split}
D_C &= 1 + w_{style} + w_{tags}\\
&w_{\text{style}} =  \left\{
  \begin{array}{l l}
    0.50 & \quad \text{\textgreater 75\% content in left two thirds}\\
    0    & \quad \text{otherwise}
  \end{array} \right.\\
&w_{\text{tags}} =  \left\{
  \begin{array}{l l}
    0.50 & \quad \text{tags in DOM but not CSS}\\
    0    & \quad \text{otherwise}
  \end{array} \right.
\end{split}
\end{equation}


Traditional Web design (and particularly design enabled by style sheets) evenly distributes content across each of the vertical thirds of a page. Our intuition is that a missing important style sheet will shift content to the left of the page rather than center content in the viewport. To identify this phenomenon, we divide the PNG snapshot of a memento into vertical thirds and measure the amount of content in each third. If a style sheet is missing \emph{and} content appears to be shifted to primarily the left two-thirds, we assume the missing style sheet was important to the distribution of content on the page.

When detecting content in the PNG snapshot, we use remaining CSS files and the HTML to determine the background color of the page. We measure the number of background and non-background colored pixels, with content being the number of non-background colored pixels. The proportion of non-background colored pixels in each vertical third gives us the amount of content in each partition.

The style threshold is determined as follows:

\begin{enumerate}
  \item Determine background color
  \item Render a PNG snapshot of the page
  \item Divide PNG into vertical third partitions
  \item Calculate number of pixels of the non-background color in each third for the viewport only (we used a 1024x768 viewport) and entire page
  \item If $\le$75\% of the non-background colored pixels are in the left two thirds of the viewport, set \emph{$w_{style}=0$} in Equation \ref{cssdamage} (CSS file does not receive a weight)
  \item If $>$75\% of the non-background colored pixels are in the left two thirds of the viewport and left two thirds of the entire page and a style sheet is missing, \emph{$w_{style}=0.5$} in Equation \ref{cssdamage} (CSS file does receive a weight)
\end{enumerate}

For example, we created two mementos of the URI-R \url{http://www.pilotonline.com/} on a local server, one as it appears live (with all style sheets -- Figure \ref{withcss2}) and the other with its style sheets removed (Figure \ref{withoutcss2}). The vertical partitions extend from the top of the PNG snapshot to the bottom.
 The percent of non-background color pixels in the viewports of our mementos are shown in their respective thirds in Figures \ref{withcss2} and \ref{withoutcss2}. Notice that the non-background pixels (text, images, etc.) shift left when the CSS is missing. Intuitively, information is not meant to be displayed like the content in Figure \ref{withoutcss2}.

\begin{figure}[h!]
  \begin{center}
    \subfigure[We calculated that the non-background color is more evenly distributed between the three vertical partitions of the Pilot Online page with its style sheet included than when it is missing.]{\label{withcss2}\includegraphics[width=220px]{./imgs/withcss.png}}\\
    \subfigure[We calculated that the non-background color is most prevalent in the left-most vertical partition of the viewport of the Pilot Online page when it is missing its style sheet.]{\label{withoutcss2}\includegraphics[width=220px]{./imgs/nocss.png}}  
  \end{center}
  \label{pilotexample2}
  \caption{Missing style sheets causes content to shift left. We show the percent of content in the vertical partitions of the viewport.}
\end{figure}

When we consider content outside of the viewport (Figures \ref{withcss} and \ref{withoutcss}), we see the same shift of content to the left when style sheets are missing. However, the distribution of content in Figure \ref{withoutcss} is more evenly distributed because the content has shifted down and fills out the middle and right vertical partitions more than in Figure \ref{withoutcss2}. This is an indicator that the style sheets missing in Figures \ref{withoutcss2} and \ref{withoutcss} were important.

\begin{figure}[h!]
  \begin{center}
    \subfigure[When considering the entire page, the content of the page is distributed 33\% in the left, 26\% in the middle, and 41\% in the right partitions when the style sheet is present.]{\label{withcss}\includegraphics[width=100px]{./imgs/thePng_crop3.png}}\qquad
    %{\label{withcss}\includegraphics[width=0.2\textwidth]{./imgs/thePng_crop2.png}}\qquad
    \subfigure[When considering the entire page, the content of the page is distributed 84\% in the left, 15\% in the middle, and 1\% in the right partitions when the style sheet is missing.]{\label{withoutcss}\includegraphics[width=100px]{./imgs/thePngNOCSS_crop3.png}}  
    %{\label{withoutcss}\includegraphics[width=0.2\textwidth]{./imgs/thePngNOCSS_crop2.png}}  
  \end{center}
  \label{full}
  \caption{The left shift caused by a missing style sheet occurs throughout the entire page and is not limited to the viewport.}
\end{figure}


Along with the style threshold, the presence of tags on the page without a matching style suggests that the missing CSS contained the referenced formatting. If such tags exist without a matching style, $w_{tags}=0.5$ in Equation \ref{cssdamage}. 

\subsection{The $D_m$ Algorithm}
Embedded multimedia, images, and style sheets do not account for the entirety of a page's importance and usefulness. We assume that text, as defined by the DOM and included on the page, is available regardless of archival success and therefore does not contribute to the damage calculation.

%Text plays an important role in a page's informational importance. We determine the importance text plays on a page in relation to the amount of images on a page. We determine the number of words on the page ($NW$), and use the adage \emph{a picture is worth 1,000 words} as the word-to-image ratio $WPI$, giving us the importance of the text $D_T$.

%\begin{equation}
%\label{textdamage}
%\begin{split}
%NW &= \text{Number of Words from Tag-Stripped HTML}\\
%WPI &= \text{1,000 Words/Image (images text is ``worth'')}\\
%D_T &= \frac{NW}{WPI}
%\end{split}
%\end{equation}

%To normalize the damage rating of the page due to the weighting assigned, we take the sum of all potential damages (with weighting assigned) of all embedded resources. We then determine the base value of each memento based on this increased value and normalize the values of the embedded resources. 
%For example, if a page has two embedded resources, \emph{a} and \emph{b}, and \emph{a} is assigned a 0.5 additional importance weight, the total calculated importance will be 2.5 (\emph{a=1.5}, \emph{b=1}). The normalized value for each embedded resource will be 2/2.5=0.8. This makes \emph{a}=1.2 and \emph{b}=0.8 for a total potential damage of 2 which normalizes to 1 for the page.

In summary, Equations \ref{damageeqn} - \ref{cssdamage} are used to compute $D_m$ in the following manner:

\begin{enumerate}
  \item Load URI-M with PhantomJS
  \item Find Potential Damage $D_{m_{potential}}$ (Equation \ref{potentialdamage})
  \begin{enumerate}
	\item Determine CSS importance $D_C$ (Equation \ref{cssdamage})
	\item Determine Multimedia importance $D_{MM}$ (Equation \ref{imagedamage})
	\item Determine Image importance $D_I$ (Equation \ref{imagedamage})
  \end{enumerate}
  \item Determine proportion of unsuccessfully dereferenced embedded resources $M_m$ (Equation \ref{resources})
  \item Find Actual Damage $D_{m_{actual}}$ (same as Step 3, but with only those URI-Ms unsuccessfully dereferenced $R_m$)
  \item Determine total damage $D_m$=[0,1] (Equation \ref{actualdamage})
  \
  %\item Print report
  %\begin{enumerate}
	%\item  URI-Ms and associated damage $D_{[C, MM, I]}$=[0,1]
	%\item  Total Page Damage $D_{URI-M}$=[0,1]
	%\item  Percent resources missing $M$=[0,1]
  %\end{enumerate}
\end{enumerate}


With $D_m$ defined, we revisit the examples presented in Section \ref{example}. The values for $D_m$ and $M_m$ are listed in Table \ref{damageTable}. Note that the $D_m$ ratings are closer to our empirical human assessment of memento quality than the proportion of the embedded resources that are missing.

\begin{table}
\centering
\begin{tabular}{ c | c | c }
    \hline
    Figure & $D_m$ & $M_m$\\
    \hline
    \hline
\ref{undamaged} & \emph{0.09} & \emph{0.17}\\
\ref{missingBig} & \emph{0.41} & \emph{0.24}\\
\ref{missingLittle} & \emph{0.36} & \emph{0.29}\\
\ref{missingflood} & \emph{0.59} & \emph{0.38}\\
\ref{missingaldot} & \emph{0.003} & \emph{0.20}\\
    \hline
\end{tabular}
  \caption{$D_m$ vs $M_m$ for the images in Figure \ref{xkcdImgs}. Note $M_m \textgreater D_m$ in 2 of 5 cases.}
  \label{damageTable}
\end{table}

\subsection{Limitations of $D_m$ Calculation}
Not all pages and page construction methods can be evaluated by this algorithm. An edge case not handled by this algorithm is any page constructed with iframes. Our algorithm uses JavaScript to determine the rendered location of embedded multimedia and images. When the embedded media is in a page embedded within another page, our algorithm does not provide the accurate rendered location. For this reason, we exclude iframes from our algorithm. We also exclude missing audio-only multimedia.

While $D_m$ includes multimedia calculations, multimedia resources are rarely embedded in our mementos (only observed twice in our entire set of 45,341 URI-Ms). We observed that multimedia is often loaded by JavaScript files embedded in the document object model (DOM); this prevents the multimedia files from being archived since archival crawlers (at the time of this experiment) do not execute client-side JavaScript and therefore do not discover the requested files. 

Further, the JavaScript files may not operate properly when archived \cite{robustifyBlog} and may not issue a request for the target multimedia files. If the JavaScript operates properly and makes an HTTP GET request, the multimedia file would be missing (since it is not archived) and we would observe more missing embedded multimedia files. We discuss this issue further in Section \ref{pjsvhtrix}. % (similar to the URI-M \url{http://web.archive.org/web/20131211191109js_/http://pilotonline.com/sites/all/themes/hamptonroads/js/jquery.youtube.player.js}).

The $D_m$ measurement and its constituent weights was constructed by archivists as an improvement to the metric $M_m$ currently used for archive quality assurance. We do not assert that $D_m$ is a perfect measure, but rather an improvement that will require additional investigation and re-weighting to reach perfect agreement with turker evaluation. We recognize that $D_m$ should be more finely tuned to more accurately reflect turker opinion of damage.  We also avoid defining a threshold for damage acceptance; this is left to the discretion of the archivist utilizing $D_m$ to measure damage in an archive.

\subsection{Turker Assessment of $D_m$}
\label{turkerDm}


We compared $D_m$ to turker assessment and to $M_m$. As shown in Table \ref{m2table}, $D_m$ agrees with turker assessment of damage 32\% of the time, an increase of 18\% over $M_m$. Additionally, 49\% tie with a 3-2 or 2-3 split and only 16\% of the turker evaluations disagreed with the $D_m$ measure. Turkers agree more consistently when {$\Delta D_m$} is larger. If we only consider {$\Delta D_m \textgreater$} 0.30, the turkers agree with $D_m$ 71\% of the time. However with {$\Delta M_m \textgreater$} 0.30, the turkers agree only 20\% of the time. 


\begin{table}
\begin{tabular}{ c | c | c | c | c | c | c || c}
    {$\Delta D_m$} &  \multicolumn{6}{c}{Splits}\\
  & 5-0 & 4-1 & 3-2 & 2-3 & 1-4 & 0-5 & Total\\
\hline
1.0 &  &  &  &  &  & & 0.00\\
0.9 &  & 1 &  &  &  & & 0.01\\
0.8 &  &  &  &  &  & & 0.00\\
0.7 &  &  &  &  &  & & 0.00\\
0.6 &  &  & 1 &  &  & & 0.01\\
0.5 &  &  &  &  &  & & 0.00\\
0.4 & 4 & 1 &  &  &  & & 0.05\\
0.3 & 2 & 2 & 3 &  &  & & 0.07\\
0.2 &  & 2 & 1 & 2 & 2 & 1& 0.08\\
0.1 & 4 & 16 & 27 & 15 & 12 & 3& 0.77\\
0.0 &  &  &  &  &  & & 0.00\\
\hline
Total & 0.10 & 0.22 & 0.32 & 0.17 & 0.14 & 0.04 & 1.0
\end{tabular}
  \caption{The turker evaluations of the $m_2$ vs $m_3$ comparisons when using $D_m$ as a damage measurement.}
  \label{m2table}
\end{table}



\begin{table}
\centering
\begin{tabular}{cll}
\textbf{Turker} & \multicolumn{2}{c}{ \textbf{ $D_{m}$}}                           \\
\textbf{Assesment}                  &          Select $m_{2}$             &           Select $m_{3}$             \\ \cline{2-3} 
       $m_{2}$           & \multicolumn{1}{|l}{45} & \multicolumn{1}{|l|}{32} \\ \cline{2-3} 
       $m_{3}$           & \multicolumn{1}{|l}{8} & \multicolumn{1}{|l|}{14} \\ \cline{2-3} 
\end{tabular}
  \caption{Confusion matrix of the turker assessments of the $m_2$ vs $m_3$ comparison test against $D_m$.}
  \label{m2cm}
\end{table}



%acc = (TP+TN)/(P+N)
%acc = (45+14)/(45+8+14+32)
%f1 = 2tp/(2tp + fp + fn)
%f1 = (2*45)/(2*45 + 8 + 32) 

%we see the accuracy of $m_2$ vs $m_3$ against $M_m$ is 0.46 with a harmonic mean of 0.55

From the confusion matrix in Table \ref{m2cm}, we determine that the accuracy of $D_m$ when comparing $m_2$ vs $m_3$ is 0.60, and $F_1$ = 0.69. This is an improvement of 0.14 over the accuracy of $M_m$ and an improvement over the harmonic mean of $M_m$ by 0.14, showing that $D_m$ measures damage closer to turker perception. 
We also calculated the AUC in a ROC curve for $D_m$ and compared it to $M_m$ and the performance of the $m_0$ vs $m_1$ test. As shown in Table \ref{auc2}, $D_m$ has an AUC of 0.584, an increase in 0.108 over $M_m$, showing that $D_m$ outperforms $M_m$ and is closer to the performance of $m_0$ vs $m_1$ (AUC=0.789).


\begin{table}
\centering
\begin{tabular}{ c | c | c | c }
    Damage Calculation &  AUC & $F_1$ & Accuracy\\
\hline
  $M_m$ & 0.472 & 0.55 & 0.46 \\
  $D_m$ & 0.584 & 0.69 & 0.60 \\
  $M_{m_0}$ & 0.789 & 0.88  & 0.80 \\
\hline
\end{tabular}
  \caption{$D_m$ provides a closer estimate of turker perception of damage and our performance of $m_0$ vs $m_1$ than $M_m$.}
  \label{auc2}
\end{table}

\section{Damage in the Archives}
\label{eval} 

Having defined an algorithm for measuring $D_m$, we measured $D_m$ values for each of the 45,341 URI-Ms from Section \ref{turkActual}. We used these measurements to assess $D_m$'s performance relative to turker assessment and to perform damage measurements in the Internet Archive.


\subsection{Measuring the Internet Archive}
\label{missing}

With $D_m$ validated as aligning closer to turker evaluations than $M_m$, we used $D_m$ to evaluate the Internet Archive's performance. Our measurement shows that only 46\% of the 45,341 URI-Ms listed in the 1,861 TimeMaps are complete -- that is, 54\% of all URI-Ms listed in the Internet Archive TimeMaps we studied are missing at least one embedded resource\footnote{The Internet Archive performs URI canonicalization very well, and is assumed to not be a source of missing resources.}. In Figure \ref{missingByYear}, we show the average number of missing embedded resources $M_m$ along with the average calculated damage $D_m$ per URI-M per year.

%We also include vertical lines that identify major Heritrix software releases that correspond to changes in changes to the trajectory of memento damage averages across the archive.

\begin{figure}[h!]
\includegraphics[width=270px]{./imgs/missedAndDamagePerYear.png}
%\includegraphics[width=0.50\textwidth]{./imgs/occStats.png}
\caption{The average embedded resources missed per memento per year in the Internet Archive as compared to damage per memento per year ($\overline{D_m}$=0.128, $\overline{M_m}$=0.132).}
\label{missingByYear}
\end{figure}


Because the number of missed mementos is important to $M_m$ and $D_m$, we investigated the occurrence of missing and successfully dereferenced embedded resources. Most mementos are missing very few embedded resources with most missing 1-10 embedded resources (as a histogram and Cumulative Distribution Function (CDF) in Figures \ref{missingDistroHist} and \ref{missingDistroCdf}), ($\mu=1.7$, $\sigma=4.6$, $median=3$). We note the long tail on this distribution; a few mementos are missing a larger amount of embedded resources (maximum is 116). We calculate that 61\% of mementos are missing 3 or fewer embedded resources, and 85\% of mementos are missing 6 or fewer embedded resources. Most mementos have very few embedded resources ($\mu=17.6$, $\sigma=86$, $median=7$), as shown in Figures \ref{foundDistroHist} and \ref{foundDistroCdf}. A few mementos have a very large number of embedded resources (maximum is 552).

\begin{figure*}
  \begin{center}
    \subfigure[Occurrences of missing embedded resource numbers in the Internet Archive as a histogram.]{\label{missingDistroHist}\includegraphics[width=0.45\textwidth]{./imgs/occStats.png}}\qquad
    \subfigure[Distribution of missing embedded resources within the collection of Internet Archive mementos as a CDF.]{\label{missingDistroCdf}\includegraphics[width=0.45\textwidth]{./imgs/OccStatsCDF_IA.png}}  \\
    \subfigure[Occurrences of successfully dereferenced embedded resource numbers in the Internet Archive as a histogram.]{\label{foundDistroHist}\includegraphics[width=0.45\textwidth]{./imgs/occStatsFound.png}}\qquad
    \subfigure[Distribution of successfully dereferenced embedded resources within the collection of Internet Archive mementos as a CDF.]{\label{foundDistroCdf}\includegraphics[width=0.45\textwidth]{./imgs/OccStatsFoundCDF_IA.png}}  
  \end{center}
  \label{missingDistro}
  \caption{The distribution of the number of successfully dereferenced and missing embedded resources per URI-M in the Internet Archive. Note that we limited the figures to 100 missing or successfully dereferenced embedded resources, respectively.}
\end{figure*}

%\begin{figure}[h!]
%\includegraphics[width=270px]{./imgs/occStats.png}
%\includegraphics[width=270px]{./imgs/OccStatsCDF_IA.png}
%\caption{The distribution of the number of missing embedded resources per URI-M.}
%\label{missingDistro}
%\end{figure}

%\begin{figure}[h!]
%\includegraphics[width=270px]{./imgs/occStatsFound.png}
%\includegraphics[width=270px]{./imgs/OccStatsFoundCDF_IA.png}
%\caption{The number of successfully dereferenced resources is more evenly distributed than those missing (Figure \ref{missingDistro}).}
%\label{foundDistro}
%\end{figure}


%\begin{figure}[h!]
%  \begin{center}
%    \subfigure[Occurrences of successfully dereferenced embedded resource numbers as a histogram.]{\label{foundDistroHist}\includegraphics[width=0.50\textwidth]{./imgs/occStatsFound.png}}\\
%    \subfigure[Distribution of successfully dereferenced embedded resources within the collection of mementos as a CDF.]{\label{foundDistroCdf}\includegraphics[width=0.50\textwidth]{./imgs/OccStatsFoundCDF_IA.png}}  
%  \end{center}
%  \label{foundDistro}
%  \caption{The number of successfully dereferenced resources is more evenly distributed than those missing (Figure %\ref{missingDistro}).}
%\end{figure}

%\subsection{Measured Damage}
%\label{archiveDamage}


In aggregate, we observed that 45,009 of 292,192 embedded resources were missing, meaning 15\% of the embedded resources in the dataset are missing. Of these, 25,848  (57\% of the missing URI-Ms) were important, meaning they were assigned an additional weight by $D_m$ (Equations 5 and 6). The average damage of all measured mementos was 0.132.

The yearly $\overline{D_m}$ goes from 0.16 in 1998 to 0.13 in 2013. That means the Internet Archive is doing a better job (over time) reducing the total memento damage in its collection. However, the number of missing \emph{important} resources (resources with an importance $\textgreater 1$ due to added weights) is increasing, going from an average of 1.30 important resources per memento in 1997 to 2.38 important resources per memento in 2013 for an average of 2.05 missing per memento. Meanwhile, the number of unimportant missing embedded resources (damage rating weight $\leq 1$) per memento is increasing at a lesser rate, going from 1.35 in 1997 to 1.64 in 2013. This suggests that while the Internet Archive is getting better overall at mitigating damage as much as possible, the archive is missing an increasing number of embedded resources deemed important. 

The distribution of file types missing per memento (Figure \ref{occstats}) shows that most URI-Ms are missing $\ge 1$ embedded resource and that style sheets and JavaScript files are missing at higher rates over time. Missing JavaScript may lead to additional missing files (such as multimedia). Images are missing at varying rates per memento over time.

\begin{figure}[h!]
%\includegraphics[width=0.5\textwidth]{./imgs/numMissedYearlyPerMem.png}
\includegraphics[width=0.5\textwidth]{./imgs/fileTypes.png}
\caption{The number of missed embedded resources per Internet Archive memento per year and file type.
%The average is given as the red line, and the standard deviation is the pink area around the average line.
}
\label{occstats}
\end{figure}



\subsection{Measuring WebCite}
\label{webcite}

In an effort to measure a less prominent and different type of archive, we used the damage algorithm to determine $M_m$ and $D_m$ of WebCite\footnote{\url{http://webcitation.org/}}. WebCite \cite{webcite} is different from the Internet Archive's Heritrix crawler in that it is a page-at-a-time (i.e., crawls a single URI-R and not an entire site) archiving tool that creates mementos upon user request. 

Web crawlers like Heritrix operate by starting with a finite set of seed URI-Rs in a frontier -- or list of crawl targets -- and add to the frontier by extracting embedded URIs in the representations of the URI-R. This allows archival crawlers to discover embedded resources as well as new URI-Rs to crawl while creating mementos. 

The Internet Archive follows this model with the goal to archive the Web using the Heritrix crawler, while WebCite and other page-at-a-time archivers allow users to submit URI-Rs for archiving, and WebCite immediately archives the resource\footnote{The Internet Archive has recently added an \emph{on-demand} archiving utility at \url{http://archive.org/web/} under the heading ``Save Page Now'' \cite{savePage}.}. When using a page-at-a-time archival service, the resulting memento contains embedded resources with the same archival datetime \cite{temporalCoherence}. This section identifies our damage measurement of this page-at-a-time archiver and outlines the differences between Heritrix and WebCite. 

Our WebCite dataset has 992 mementos in 285 TimeMaps of our collection of 1,861 URI-Rs. The earliest available memento is from 2007, and the latest is from 2014. Only six mementos are available from 2014; therefore, we will focus on 2007-2013 as the target years of investigation due to the limited number of 2014 mementos, as well as to match the period of observation of the Internet Archive. The $\overline{D_m}$ of the collection over all years is $0.397$ ($\sigma=0.194$), and the $\overline{M_m}$ is $0.176$ ($\sigma=0.0926$). All of the mementos in this collection are missing at least one embedded resource -- 100\% of the mementos are incomplete. 

As shown in Figure \ref{missingByYearWC}, the $\overline{D_m}$ in WebCite is increasing over time, going from 0.285 in 2007 to 0.442 in 2013. Meanwhile, the average $M_m$ remains steady, going from 0.135 in 2007 to 0.139 in 2013. Only slight variation occurs, peaking at 0.287 in 2010. 

Compared to the Internet Archive, WebCite has a higher damage value as well as is missing a larger percentage of embedded resources. Additionally, $D_m$ per memento is higher, indicating that a larger percentage of missing embedded resources are important (3,514 or 41.7\%) in WebCite than in the Internet Archive.

\begin{figure}[h!]
\includegraphics[width=270px]{./imgs/MissedAndDamagePerYear_webcite.png}
%\includegraphics[width=0.50\textwidth]{./imgs/occStats.png}
\caption{The average embedded resources missed per memento per year in WebCite as compared to damage per memento per year ($\overline{D_m}$=0.397, $\overline{M_m}$=0.176).}
\label{missingByYearWC}
\end{figure}

WebCite is missing on average 10.1 embedded resources per memento ($\sigma=8.0$, $median=2$). This distribution exhibits a long tail, with a few mementos missing a large number of embedded resources (maximum is 133). WebCite mementos successfully dereference on average 15.3 embedded resources per memento ($\sigma=30.7$, $median=4$); again note the long tail (maximum is 154). Across the entire collection, 8,420 of 54,824, or 15.4\% of the embedded resources were missing in our investigation. We calculate that 56\% of mementos are missing 3 or fewer embedded resources, and 74\% of mementos are missing 6 or fewer embedded resources (Figure 9).

%\begin{figure}[h!]
%\includegraphics[width=270px]{./imgs/OccStats_webcite.png}
%\includegraphics[width=270px]{./imgs/OccStatsCDF.png}
%\caption{The distribution of the number of missing embedded resources per URI-M in WebCite.}
%\label{missingDistroWC}
%\end{figure}

    
\begin{figure*}
  \begin{center}
    \subfigure[Occurrences of missing embedded resource numbers in WebCite as a histogram.]{\label{missingDistroWCHisto}\includegraphics[width=0.45\textwidth]{./imgs/OccStats_webcite.png}}\quad
    \subfigure[Distribution of missing embedded resources within the collection of WebCite mementos as a CDF.]{\label{missingDistroWCCdf}\includegraphics[width=0.45\textwidth]{./imgs/occStatsCDF09.png}}  \\
    \subfigure[Occurrences of successfully dereferenced embedded resource numbers in WebCite as a histogram.]{\label{foundDistroWCHisto}\includegraphics[width=0.45\textwidth]{./imgs/OccStatsFound_webcite.png}}\quad
    \subfigure[Distribution of successfully dereferenced embedded resources within the collection of WebCite mementos as a CDF.]{\label{foundDistroWCCdf}\includegraphics[width=0.45\textwidth]{./imgs/occStatsFoundCDF09.png}}  
  \end{center}
  \label{missingDistroWC}
  \caption{The distribution of the number of successfully dereferenced and missing embedded resources per URI-M in WebCite. Note that we limited the figures to 60 missing or successfully dereferenced embedded resources, respectively.}
\end{figure*}


%\begin{figure}[h!]
%\includegraphics[width=270px]{./imgs/OccStatsFound_webcite.png}
%\includegraphics[width=270px]{./imgs/OccStatsFoundCDF.png}
%\caption{In WebCite, the number of successfully dereferenced resources is more evenly distributed than those missing (Figure %\ref{missingDistroWC}).}
%\label{foundDistroWC}
%\end{figure}

%\begin{figure}[h!]
%  \begin{center}
%    \subfigure[Occurrences of successfully derefernced embedded resource numbers in WebCite as a histogram.]{\label{foundDistroWCHisto}\includegraphics[width=0.50\textwidth]{./imgs/OccStatsFound_webcite.png}}\\
%    \subfigure[Distribution of successfully derefernced embedded resources within the collection of WebCite mementos as a CDF.]{\label{foundDistroWCCdf}\includegraphics[width=0.50\textwidth]{./imgs/occStatsFoundCDF09.png}}  
%  \end{center}
%  \label{foundDistroWC}
%  \caption{The distribution of the number of successfully dereferenced embedded resources per URI-M in WebCite.}
%\end{figure}

%\subsection{Measured Damage}
%\label{archiveDamage}


The distribution of file types missing per memento (Figure \ref{occstatsWC}) shows that most URI-Ms are missing $\ge 1$ embedded image and CSS resources, on average. WebCite has a lower occurrence of missing style sheets, but a higher occurrence of missing images. %This will impact future work with $D_m$. If we change the weighting of the importance of missing embedded resources -- if we weight missing CSS as having a higher impact on overall $D_m$, WebCite's collection might have a lower average $D_m$. However, more investigation is needed before this conclusion may be reached. 

Our previous investigation showed that WebCite has difficulties when encountering JavaScript and embedded iframes \cite{ijdl}. However, its archiving policies provide immediate results as opposed to crawlers that may incur a delay between the time a URI-R is added to the frontier and a memento is created. WebCite's difficulties with JavaScript may contribute to the missing embedded resources if they were loaded through JavaScript.


\begin{figure}[h!]
%\includegraphics[width=0.5\textwidth]{./imgs/numMissedYearlyPerMem.png}
\includegraphics[width=0.5\textwidth]{./imgs/FileTypes_webcite.png}
\caption{The number of missed embedded resources per WebCite memento per year and file type.
%The average is given as the red line, and the standard deviation is the pink area around the average line.
}
\label{occstatsWC}
\end{figure}


\section{Impact of JavaScript on Damage}
\label{wcjs}


As a preliminary investigation of the impact of JavaScript on archival tools, we set up an experiment to use Heritrix and PhantomJS to crawl the same set of URI-Rs and measure the damage difference between the two sets of mementos. Our goal is to understand how $D_m$ is impacted by JavaScript by comparing mementos archived by a crawler that can execute JavaScript (PhantomJS) and a crawler that does not execute JavaScript (Heritrix).

\subsection{PhantomJS vs Heritrix}
\label{pjsvhtrix}
Representations of Web resources are increasingly reliant on JavaScript and other client-side technologies to load embedded resources and control the activity on the client or request additional data or resources (e.g., via Ajax) after the initial page load. We refer to representations that are changed by client-side code, such as JavaScript, as \emph{deferred representations} because the full representation is not realized until after the initial page load. 
Crawlers are unable to discover the resources requested via Ajax and are missing embedded resources which ultimately causes the mementos of the crawled resources with deferred representations to be incomplete and have higher $D_m$. 

To mitigate the impact web developers' practice of using JavaScript and Ajax to load embedded resources, crawlers like Heritrix have constructed approaches for extracting links from embedded JavaScript to be added to crawl frontiers  \cite{htrixJS} (most recently, Google's crawler \cite{googleJS}). Archive-It has recently adopted Umbra to archive a hand-selected set of URI-Rs known to have deferred representations \cite{umbra}. However, this is not a solution to the challenges that JavaScript introduces in the archives, but is a mitigation of the impact for a small set of URI-Rs (e.g., facebook.com, twitter.com URI-Rs).

Because archival crawlers' abilities differ from the abilities of browsers, the archives currently hold a representation of the Web from the point of view of crawlers and not Web users. That is, what we archive is increasingly different than what users experience.

\subsection{Crawling Deferred Representations}
\label{crawlDeferred}
We sampled 50 URI-Rs by randomly generating Bitly.com URIs and identifying the URI-Rs to which the bitly URIs redirected. We then classified the 50 URI-Rs as having deferred representations and crawled the set of URIs with Heritrix and PhantomJS. 

During the Heritrix crawl, we used the 50 URI-Rs as a set of seed URIs and allowed Heritrix to create their mementos. The final frontier size of this crawl was 1,588 URIs of embedded resources used to create the mementos. Using our damage algorithm, we measured the damage of the mementos created by Heritrix and found that $\overline{D_m}$=0.148. Recall that for the Internet Archive, $\overline{D_m}$=0.13.

To ensure the crawler executes JavaScript and captures JavaScript-dependent resources during the creation of mementos, we then crawled the 50 URI-Rs with PhantomJS. We recorded the embedded resources needed to create the representation, including those originating from JavaScript. This created a frontier of 3,364 URIs which we used as a seed URI list in Heritrix. We then used Heritrix to create the mementos using only the seed URI list, effectively creating mementos using the frontier list of PhantomJS. For this crawl, $\overline{D_m}$=0.1291. 

PhantomJS provided a 13.5\% improvement to the collection damage over Heritrix. This provides further evidence that JavaScript-dependent representations reduce the quality of mementos due to traditional crawlers' inability to execute JavaScript.

Not only does using PhantomJS provide a larger crawl frontier, but the damage rating of the resulting mementos is lower. In short, this initial investigation suggests that using PhantomJS mitigates the impact of JavaScript on resources with deferred representations and results in higher-quality mementos.

\section{Measuring Archive.today}
\label{archivetoday}
Archive.today \cite{archivetoday} is another page-at-a-time archival service like WebCite. Archive.today and WebCite were established for different purposes, each offering its own benefits. WebCite was established for the purpose archiving pages that appear in scholary publications \cite{kleinLinkRot}, although its use has since expanded to the general Web.  Archive.today was established later and with a more modern technology base with respect to JavaScript and Ajax, and always had a focus on the general Web user. Archive.today does not archive resources such as PDFs or XML, while WebCite makes an attempt to archive such resources. 

While WebCite does not properly archive deferred representations, Archive.today creates mementos that limit leakage \cite{zombies, archiveisblog} (leakage occurs when a memento improperly embeds live Web resources, often through JavaScript) and missing embedded resources typically occurring in other archival services that ignore JavaScript. We leave the decision as to which service is better under what conditions as an exercise for the reader. However, in an effort to study the impact of Archive.today's handling of JavaScript on $D_m$, we submitted each of our 1,861 URI-Rs to Archive.today for archiving to create mementos of each resource. 

When Archive.today creates a memento, it modifies the DOM to remove references to embedded resources that were not available at archive time (i.e., embedded resources that returned a non-200 HTTP response code) \cite{refreshZombies}. This results in a memento that -- if created properly -- has no missing embedded resources. Additionally, Archive.today obfuscates the URIs of embedded mementos, preventing a reliable mapping from URI-R to URI-M. For example, the live resource might have an embedded image such as

\begin{verbatim}
<img src="http://d3n8a8pro7vhmx.cloudfront.
	net/peoplesgrocery/sites/1/meta_images/
	large/pg_sidebar_logo.png?1372698696" 
	width="315" height="315"> 
\end{verbatim}

\noindent and Archive.today will convert the URI-R to the following URI-M:

\begin{verbatim}
<img src="https://archive.today/v9cDq/
	e632aee8994b72b42e8f7a977ddc1cb63329d9f5.png" 
	style="text-align:left;box-sizing: border-box;
	... "/>
\end{verbatim}

Due to these two archival practices, the damage algorithm used in this paper is ineffective for determining memento quality. For this reason, we alter the method of measuring the effectiveness of Archive.today's archival process. 

We initiated the archiving of each URI-R in our collection by Archive.today. We counted the number of embedded resources that were successfully loaded into the live resources (i.e., returned an HTTP 200 response when their URIs were dereferenced) and compared this number to the number of embedded resources successfully archived by Archive.today, resulting in a delta between live resource and memento embedded resources that we will refer to as $\Delta_m$. It is worth noting that the delta between the number of embedded resources in live resources and mementos ($\Delta_m$) is a measure of neither $M_m$ nor $D_m$, but is instead a mechanism for understanding memento fidelity. 

We found that Archive.today has a $\overline{\Delta_m}$=19.9 ($\sigma=39.2$), meaning that on average, Archive.today did not archive 19.9 embedded resources from the live page due to either its inability to archive the resources, or because Archive.today may have deemed the embedded resources not suitable for archiving\footnote{Archive.today lists the resources it saves and does not save in its FAQ page at \url{http://archive.today/faq.html}}. A histogram of all $\Delta_m$ measures is provided in Figure \ref{atHisto}.



\begin{figure}[h!]
  \begin{center}
    \subfigure[Histogram of the memento vs live resource $\Delta_m$ in Archive.today.]{\label{atHisto}\includegraphics[width=0.5\textwidth]{./imgs/AtHisto.png}}\\
    \subfigure[Histogram of the memento vs live resource $\Delta_m$ in WebCite.]{\label{wcHisto}\includegraphics[width=0.5\textwidth]{./imgs/WcHisto.png}}  
  \end{center}
  \label{deltaHisto}
  \caption{The $\Delta_m$ measurements of Archive.today and WebCite indicate that Archive.today creates higher fidelity mementos than WebCite.}
\end{figure}


%\begin{figure}[h!]
%\includegraphics[width=270px]{./imgs/atHisto.png}
%\caption{Histogram of the memento vs live resource $\Delta_m$ in Archive.today.}
%\label{athisto}
%\end{figure}

We submitted each URI-R in the collection to WebCite and recorded $\Delta_m$ for the WebCite mementos in the exact way we measured $\Delta_m$ for Archive.today. In this way, we can compare the two page-at-a-time archivers to determine which service creates higher fidelity mementos. WebCite has a $\overline{\Delta_m}$=21.6 ($\sigma=41.7$), which is higher than the $\overline{\Delta_m}$ of Archive.today. The histogram of the WebCite $\Delta_m$ is provided in Figure \ref{wcHisto}. The higher WebCite $\overline{\Delta_m}$ indicates that Archive.today creates higher fidelity mementos than WebCite, likely due to its superior support of JavaScript dependent representations.

%\begin{figure}[h!]
%\includegraphics[width=270px]{./imgs/wcHisto.png}
%\caption{Histogram of the memento vs live resource $\Delta_m$ in WebCite.}
%\label{wchisto}
%\end{figure}


\section{Conclusions}
\label{conclusion}
In this paper, we demonstrated that Web users (as represented by Mechanical Turk Workers) can correctly identify undamaged mementos\footnote{``Undamaged'' mementos are mementos without purposefully removed embedded resources. Note that some live Web resources may have damage because they are missing embedded resources, and this damage is reflected in our undamaged and subsequently intentionally damaged mementos.} ($m_0$ vs $m_1$) 81\% of the time when presented with an original and a manually damaged pair of mementos. After randomly selecting 100 URI-Ms from the Internet Archive TimeMaps of 1,861 URI-Rs, we show that turkers' assessment of damage does not match that of $M_m$; in fact, their perception of damage more closely aligns with a random selection than with $M_m$. 

To provide a damage metric closer to the perception of Web users, we proposed $D_m$, a damage calculation algorithm that estimates embedded resource importance to determine the perceived damage of mementos. Using turker evaluations, we showed that $D_m$ aligns with turker perception 32\% of the time when considering all {$\Delta D_m$} values -- an improvement of 17\% over $M_m$. If we limit {$\Delta D_m \textgreater$} 0.30, we achieve an agreement of 71\%, an improvement of 51\% over $M_m$. We show that the performance of $D_m$ is closer to that of the $m_0$ vs $m_1$ test than both $M_m$ and a random selection.

We used $D_m$ to measure the performance of the Internet Archive by measuring $\overline{D_m}$ of 1,861 URI-Rs. The average damage of the Internet Archive collection is 0.13 per memento and is missing 15\% of its embedded resources. Mementos are missing 2.05 important resources on average. The Internet Archive has gotten better at mitigating damage over time, reducing $D_m$ from 0.16 (1998) to 0.13 (2013). 

Page-at-a-time archivers perform differently than the Internet Archive. We measured mementos of our collection in WebCite, finding that the average damage of the collection is 0.397 per memento and is missing 18\% of its embedded resources. Mementos are missing 10.1 resources on average. Even though damage in the Internet Archive is improving, the damage in WebCite is getting worse, increasing $D_m$ from 0.375 (2007) to 0.475 (2013). 

We also demonstrate that JavaScript-dependent representations have a detrimental impact on $D_m$ and $M_m$. By using a crawl strategy in which JavaScript is executed during the crawl, damage in the resulting mementos can be reduced by 13.5\%.

With $D_m$, archival services can evaluate their performance and the quality of their mementos. The archives could measure a selection of mementos (either randomly sampled or by identifying those missing a proportion of embedded resources, such as {$\Delta D_m \textgreater$} 0.30) for damage to determine whether or not they have been satisfactorily archived. That is, with this algorithm, the archives can provide the greatest damage improvement through targeted repair efforts (e.g., identify mementos that require additional attention to ensure proper archiving). Archives can also use historical damage ratings of a URI-R to identify memento improvements or changes.

We also measured the damage of mementos in WebCite, and demonstrated that the damage in the Internet Archive ($\overline{D_m}$=0.128) is less than that in WebCite ($\overline{D_m}$=0.397). We know from previous works that WebCite does not archive JavaScript-dependent representations easily. We also measured Archive.today to determine the fidelity of an archival service that makes an effort to use headless browsing to capture JavaScript dependent representations. We found that Archive.today had a delta of 19.9 embedded resources between mementos and live resources, while WebCite had a delta of 21.6. This shows that Archive.today provides a higher fidelity memento than WebCite.

This is a preliminary investigation of memento damage. We have shown that percentage of embedded resources missing is not an accurate representation of damage and have proposed a more accurate metric.  Our future work will continue to improve upon the metric by using larger datasets, more turkers, and machine learning to further hone $D_m$. This will include a refinement of the relative weights of the embedded resources (e.g., the relative importance of CSS vs. images). We will also investigate the cumulative damage rating over time. For example, a logo that never changes over a 5 year period could have increased importance due to its use over multiple mementos. We plan to also measure the damage improvement of mementos if embedded resources are retroactively captured and included in past mementos. This cumulative damage improvement can help identify embedded resources that should be targeted by archives.


\section{Acknowledgments}
This work supported in part by the National Science Foundation (NSF) (IIS 1009392), the Library of Congress, and the National Endowment for the Humanities (NEH) Digital Humanities Implementation Grant (DHIG) (HK-50181-14).


%\bibliographystyle{abbrv}
\bibliographystyle{spmpsci}
%\bibliography{_mybibtex} 
\bibliography{_mybibtex}  





\end{document}
% end of file template.tex

